\documentclass[12pt]{article}

\usepackage{hyperref} % гиперссылки

\usepackage{tikz} % картинки в tikz
\usepackage{microtype} % свешивание пунктуации

\usepackage{array} % для столбцов фиксированной ширины

\usepackage{indentfirst} % отступ в первом параграфе

\usepackage{sectsty} % для центрирования названий частей
\allsectionsfont{\centering}

\usepackage{amsmath} % куча стандартных математических плюшек

\usepackage{comment} % добавление длинных комментариев

\usepackage[top=2cm, left=1.2cm, right=1.2cm, bottom=2cm]{geometry} % размер текста на странице

\usepackage{lastpage} % чтобы узнать номер последней страницы

\usepackage{enumitem} % дополнительные плюшки для списков
%  например \begin{enumerate}[resume] позволяет продолжить нумерацию в новом списке

\usepackage{caption} % что-то делает с подписями рисунков :)


\usepackage{fancyhdr} % весёлые колонтитулы
\pagestyle{fancy}
\lhead{Амбициозный план семинаров по теории вероятностей}
\chead{}
\rhead{2018-2019}
\lfoot{}
\cfoot{}
\rfoot{\thepage/\pageref{LastPage}}
\renewcommand{\headrulewidth}{0.4pt}
\renewcommand{\footrulewidth}{0.4pt}



\usepackage{todonotes} % для вставки в документ заметок о том, что осталось сделать
% \todo{Здесь надо коэффициенты исправить}
% \missingfigure{Здесь будет Последний день Помпеи}
% \listoftodos — печатает все поставленные \todo'шки



\usepackage{booktabs} % красивые таблицы
% заповеди из докупентации:
% 1. Не используйте вертикальные линни
% 2. Не используйте двойные линии
% 3. Единицы измерения - в шапку таблицы
% 4. Не сокращайте .1 вместо 0.1
% 5. Повторяющееся значение повторяйте, а не говорите "то же"



\usepackage{fontspec} % что-то про шрифты?
\usepackage{polyglossia} % русификация xelatex

\setmainlanguage{russian}
\setotherlanguages{english}

% download "Linux Libertine" fonts:
% http://www.linuxlibertine.org/index.php?id=91&L=1
\setmainfont{Linux Libertine O} % or Helvetica, Arial, Cambria
% why do we need \newfontfamily:
% http://tex.stackexchange.com/questions/91507/
\newfontfamily{\cyrillicfonttt}{Linux Libertine O}

\AddEnumerateCounter{\asbuk}{\russian@alph}{щ} % для списков с русскими буквами
\setlist[enumerate, 2]{label=\asbuk*),ref=\asbuk*}

%% эконометрические сокращения
\DeclareMathOperator{\Cov}{Cov}
\DeclareMathOperator{\Corr}{Corr}
\DeclareMathOperator{\Var}{Var}
\DeclareMathOperator{\E}{E}
\def \hb{\hat{\beta}}
\def \hs{\hat{\sigma}}
\def \htheta{\hat{\theta}}
\def \s{\sigma}
\def \hy{\hat{y}}
\def \hY{\hat{Y}}
\def \v1{\vec{1}}
\def \e{\varepsilon}
\def \he{\hat{\e}}
\def \z{z}
\def \hVar{\widehat{\Var}}
\def \hCorr{\widehat{\Corr}}
\def \hCov{\widehat{\Cov}}
\def \cN{\mathcal{N}}



\usepackage[bibencoding = auto,
backend = biber,
sorting = none,
style=alphabetic]{biblatex}

\addbibresource{em1_pset_v2.bib}



% делаем короче интервал в списках
\setlength{\itemsep}{0pt}
\setlength{\parskip}{0pt}
\setlength{\parsep}{0pt}


\begin{document}

Сентябрь: вечеринка установки софта!


\begin{enumerate}
  \item Вперёд в рукопашную: события, случайные величины. Вероятность и ожидание в дискретном случае. Дерево.
  \item Метод первого шага. Рекуррентные уравнения на вероятность и ожидание.
  \item Условная вероятность. Таблица сопряжённости. Парадокс Симпсона.
  \item Задачи на условную вероятность и метод первого шага.
  \item Дисперсия. Разложение случайной величины в сумму. Аддитивность ожидания.
  \item Функция плотности и вероятностная дифференциальная форма.
  \item Рождение распределений: от Бернулли до экспоненциального и Пуассона.
  \item Прогулки под вершинами.
\end{enumerate}

Контрольная 1.

\begin{enumerate}[resume]
  \item Дискретное совместное распределение. Ковариации и корреляции.
  \item Условное математическое ожидание. Функция и случайная величина.
  \item Совместная функция плотности. Ковариации в непрерывном случае.
  \item Долой неравенство Йенсена, Чебышёва и Маркова.
  \item Полный беспредел: ЗБЧ и сходимость по вероятностям.
  \item Рождение нормального распределения.
  \item Работа с нормальным распределением. ЦПТ.
\end{enumerate}

При везении:

\begin{itemize}
  \item геометрический смысл дисперсии, ковариации, корреляции
  \item успеть пуассоновский поток
  \item рядом с дисперсией рассказать энтропию
  \item начать байесовский подход сейчас, а не в 4м модуле
  \item Симуляции R/python/julia
  \item Перенести рождение нормального распределения в многомерное?
  \item рассказать про дифференциальные уравнения методом Крофтона
\end{itemize}


Контрольная 2.

Промежуточный экзамен.

\begin{enumerate}[resume]
  \item Многомерное нормальное распределение.
  \item Выборочные характеристики.
  \item Максимально правдоподобно!
  \item Метод моментов: один момент!
  \item Свойства оценок: несмещённость, состоятельность, эффективность.
  \item Информация Фишера. Неравенство Крамера-Рао для проверки эффективности.
  \item Дисперсия и распределение ML оценки.
  \item Дельта метод.
  \item Проекции и выборочные характеристики.
  \item Проекции и хи-квадрат распределение.
  \item Распределение Стьюдента и Фишера. С доказательствами!
\end{enumerate}

Контрольная 3.

\begin{enumerate}[resume]
  \item Доверительный интервал: общий случай и для ML-оценки.
  \item Исторический интервал для математического ожидании и тяжёлой доли.
  \item Проверка гипотез в общем случае. Уровень значимости.
  \item Три классических теста. LR-тест.
  \item Три классических теста. LM-тест и тест Вальда.
  \item Тесты на таблицы сопряжённости: LR-тест и тест Пирсона.
  \item Тесты и доверительные интервалы на разницу ожиданий и дисперсию.
  \item Непараметрические тесты: Манн-Уитни и Колмогоров-Смирнов.
  \item Байесовский подход. Явный вывод апостериорной плотности.
  \item Байесовский подход. Алгоритмы MCMC.
\end{enumerate}

Контрольная 4.

Финальный экзамен.

При везении:

\begin{itemize}
  \item бутстрэп
  \item меньше готовых тестов и больше идей
  \item практика MCMC через rethinking + brms или pymc3/4?
\end{itemize}


\end{document}
